
% macro pour le 'debuggage': permet de corriger le document avec plus de facilit�
\newcommand{\mydebuglabel}[1]{\label{#1}}
%\newcommand{\debuglabel}[1]{\label{#1}}

% nouveaux environements de theoreme
%----------------------
\newtheorem{proposition}{Proposition}
%\newtheorem{lemme}{Lemme}
\newtheorem{exemple}{Exemple}
%\newcommand{\mydefinition}[2]{\begin{definition}{\bf #1:} #2$\diamond$\end{definition}}
%\newcommand{\mydefinition}[1]{\begin{definition}#1\end{definitio\newtheorem{theoreme}{Theorem}[section]
\newcommand{\deftitle}[1]{\begin{definition}{\bf (#1)}}

% alias de notation mathematiques
%-----------------------
%%%% debut macro %%%% %\newcommand{\barre}[1]{#1}  %% pour ajouter une barre horizontale au dessus d'un symbole

%\newcommand{\vector}[1]
%{\mathchoice
%{\overset{\mbox{\xymatrix{*{\hphantom{\displaystyle #1}}
%\ar[]+L;[]+R}}}{\displaystyle #1}}%
%{\overset{\mbox{\xymatrix{*{\hphantom{\textstyle #1}}
%\ar[]+L;[]+R}}}{\textstyle #1}}%
%{\overset{\mbox{\xymatrix{*{\hphantom{\scriptstyle #1}}
%\ar[]+L;[]+R}}}{\scriptstyle #1}}%
%{\overset{\mbox{\xymatrix{*{\hphantom{\scriptscriptstyle #1}}
%\ar[]+L;[]+R}}}{\scriptscriptstyle #1}}% 
%}

\newcommand{\barre}[1]{\overline{#1}}%% pour ajouter une barre horizontale au dessus d'un symbole
\newcommand{\prosite}[1]{{\bf{\small#1}}}%% pour �crire des motifs prosite
\newcommand{\permille}{\hbox{$\,^0\!/_{00}$}}%% symbole pourmille


%Pour changer la distance de la fl�che, on peut proc�der ainsi.
%\renewcommand{\ra}[1]
%{\overset{\raisebox{-1pt}{\mbox{\xymatrix{*{\hphantom{#1}}
%\ar[]+L;[]+R}}}}{#1}}
%%%% fin macro %%%%


%% alias specifiques pour les relations, utilisables seulement en mode mathematique
%-----------------------
\newcommand{\La}{\ensuremath{\langle}}
\newcommand{\Ra}{\ensuremath{\rangle}}

\newcommand{\dcp}[2]{\parallel_p^{\La #1,#2 \Ra}}
\newcommand{\dcs}[2]{\parallel_s^{\La #1,#2 \Ra}}
\newcommand{\cp}[2]{\parallel_p^{#1,#2}}
\newcommand{\cs}[2]{\parallel_s^{#1,#2}}
\newcommand{\cscc}{\parallel_s^{c_1,c_2}} % racourci pour cs avec comme classe c1 et c2
\newcommand{\scp}{\parallel_p} % scp pour Simple Common Prefix (simple = pas de classe)
\newcommand{\scs}{\parallel_s} % scs pour Simple Common Suffix (simple = pas de classe)
%\newcommand{\inc}[2]{\not\sim^{#1,#2}}
\newcommand{\sinc}{\parallel_s^{\neq}} % sinc pour Simple Incompatible (simple = pas de classe)
\newcommand{\stc}{\scs} % stc: suffixe toute classe
\newcommand{\scce}{\parallel_s^{ce}} % scce: suffixe commun de contre exemple
%\newcommand{\cp}[2]{{\vphantom{\parallel_p^{#2}}}^{#1}{\parallel_p^{#2}}}
%\newcommand{\cs}[2]{{\vphantom{\parallel_s^{#2}}}^{#1}{\parallel_s^{#2}}}

\newcommand{\suff}[2]{\text{{\it Suff}}_{#1}(#2)}
\newcommand{\pref}[2]{\text{{\it Pref}}_{#1}(#2)}
\newcommand{\relalias}{\ifmmode{\parallel_{r}}\else{$\parallel_{r}$}\fi} % un symbol pour repr�senter une relation qque entre 2 etats

%% alias pour la definition des classes d'automates, des acceptations et des automates particuliers (en mode math ou non)
%-----------------------
\newcommand{\assurepasmath}[1]{\ifmmode\text{#1}\else #1\fi}
%\newcommand{\cnfa}{\ifmmode\text{NFAC}\else NFAC}
\newcommand{\ufa}{\assurepasmath{UFA}}
\newcommand{\ufas}{\assurepasmath{UFAs}}
\newcommand{\dfa}{\assurepasmath{DFA}}
\newcommand{\dfas}{\assurepasmath{DFAs}}
\newcommand{\nfa}{\assurepasmath{NFA}}
\newcommand{\nfas}{\assurepasmath{NFAs}}
\newcommand{\rfsa}{\assurepasmath{RFSA}}
\newcommand{\rfsas}{\assurepasmath{RFSAs}}
\newcommand{\kamb}[1]{\ifmmode\text{AFA}_{#1}\else AFA\ensuremath{_{\text{#1}}}\fi}
\newcommand{\cnfa}{\assurepasmath{{NFC}}}
\newcommand{\cnfas}{\assurepasmath{{NFCs}}}
\newcommand{\cdfa}{\assurepasmath{{DFC}}}
\newcommand{\cdfas}{\assurepasmath{{DFCs}}}
\newcommand{\cufa}{\assurepasmath{{UFC}}}
\newcommand{\cufas}{\assurepasmath{{UFCs}}}
\newcommand{\ckamb}[1]{\ifmmode\text{{AFC}}_{#1}\else {AFC}\ensuremath{_{\text{#1}}}\fi}

\newcommand{\Sa}{\ifmmode\mathcal{S}\else\ensuremath{\mathcal{S}}\fi}

\newcommand{\auset}[2]{\ifmmode\boldsymbol{D}_{#1}(#2)\else\ensuremath{\boldsymbol{D}_{\text{#1}}(\text{#2})}\fi}
\newcommand{\partitions}[1]{\ifmmode\boldsymbol{P}(#1)\else\ensuremath{\boldsymbol{P}}(#1)\fi}
% ensemble des partitions restreinte a une classe d'automate
%\newcommand{\partitionsr}[2]{\ifmmode\mathcal{P}art_{#1}(#2)\else\ensuremath{\mathcal{P}art_{#1}}(#2)\fi}
\newcommand{\partitionsr}[2]{\ifmmode\boldsymbol{P}_{#1}(#2)\else\ensuremath{\boldsymbol{P}_{#1}}(#2)\fi}

\newcommand{\Acc}[2]{\ifmmode Acc_{#1}(#2)\else\ensuremath{Acc_{\text{#1}}(\text{#2})}\fi}
\newcommand{\smca}[1]{\ensuremath{\text{{\it MCA}}(}#1\ensuremath{)}}
\newcommand{\mca}{\ensuremath{\text{{\it MCA}}}} % mca sans le sample
\newcommand{\ua}{\ensuremath{\text{{\it UA}}}}
\newcommand{\sua}[1]{\ensuremath{\text{{\it UA}}(}#1\ensuremath{)}}
\newcommand{\mcak}[1]{\ensuremath{\text{{\it MCA}}_{#1}}}
\newcommand{\smcak}[2]{\ensuremath{\text{{\it MCA}}_{#1}(}#2\ensuremath{)}}
\newcommand{\smcau}[1]{\smcak{u}{#1}}
\newcommand{\mcau}{\mcak{u}}
\newcommand{\pta}{\ensuremath{\text{{\it PTA}}}}
\newcommand{\spta}[1]{\ensuremath{\text{{\it PTA}}(}#1\ensuremath{)}}
\newcommand{\au}[1]{\ensuremath{\La\Sigma,\Gamma_{#1},Q_{#1},I_{#1},\delta_{#1},\rho_{#1}\Ra}}% automate classifieur
\newcommand{\nau}[1]{\ensuremath{\La\Sigma,Q_{#1},I_{#1},\delta_{#1},F_{#1}\Ra}}% automate normal

%% alias des op�rateurs de parcours
%---------------------------------
% seulement en mode math !!!
% fusion simple sur partition
\newcommand{\ispar}{\prec} % inf�riorit� stricte
\newcommand{\ipar}{\prec^*} % inf�riorit�
\newcommand{\fuspar}{\xrightarrow{\prec}} % op�rateur de fusion
%\newcommand{\fispar}{\xrightarrow{\succ}}% op�rateur de fission
% fusion simple sur automates
\newcommand{\isaut}{\prec_{A}} % inf�riorit� stricte
\newcommand{\iaut}{\prec^*_{A}} % inf�riorit�
\newcommand{\fusaut}{\xrightarrow{\prec_A}}% op�rateur de fusion simple
\newcommand{\fisaut}{\xrightarrow{\succ_A}}% op�rateur de fission simple
\newcommand{\fussaut}{\xrightarrow{A*}}% op�rateur de suite de fusion simples
% fusion deterministe
\newcommand{\idet}{\prec^*_{det}}
\newcommand{\isdet}{\prec_{det}}
\newcommand{\fdet}{\xrightarrow{det}}
% fusion restreinte au determinisme
\newcommand{\irdet}{\prec^*_{D}}
\newcommand{\isrdet}{\prec_{D}}
\newcommand{\frdet}{\xrightarrow{D}}
% fusion desambiguisante
\newcommand{\ides}{\prec^*_{des}}
\newcommand{\isdes}{\prec_{des}}
\newcommand{\fdes}{\xrightarrow{des}}
% fusion restreinte a la k-ambiguite
\newcommand{\ikam}[1]{\prec^*_{#1-amb}}
\newcommand{\iskam}[1]{\prec_{#1-amb}}
\newcommand{\fkam}[1]{\xrightarrow{#1-amb}}

%% alias de presentation
%-----------------------
\newcommand{\etc}{etc.}
%\newcommand{\commentaire}[1]{{\tiny #1}}
\newcommand{\commentaire}[1]{}
\newcommand{\proof}[1]{{\bf Preuve:} #1}
%\newcommand{\proof}[1]{{\bf Proof:} #1$\square$}
\newcommand{\hintproof}[1]{{\bf Id�e de la preuve:} #1}
%\newcommand{\hintproof}[1]{{\bf Hint of the proof:} #1$\square$}
%\newcommand{\mycaption}[1]{\caption{{\small #1}}}
%\newcommand{\mycaption}[1]{\caption{ #1}}

%\newcommand{\remark}[1]{{\small Remark: \emph{#1}}}
\newcommand{\remark}[1]{Remark: \emph{#1}}

\newcommand{\resumefr}[1]{%
\thispagestyle{empty}
\section*{R�sum�}
#1 
\vfill
}

% >> macro pour la separation d'un bout de page en deux parties, utile pour les figures et leur caption a droite ou gauche
% l'argument specifie la taille de la partie gauche
\newlength\jataille
\newcommand{\figgauche}[3]%
{\jataille=\textwidth\advance\jataille by -#1
\parbox{#1}{#2}
\parbox{\jataille}{#3}
}

\newcommand{\figtxt}[1]%
{{\it {\small #1}}}
% << fin macro

% >> macro pour tracer un trai horizontal sur la largeur de la page
\newcommand{\traithoriz}{\raisebox{0.4em}{\vrule depth 0pt height 0.4pt width \textwidth}}


% nom des algos
\newcommand{\edsm}{EDSM}
\newcommand{\rpni}{RPNI}
\newcommand{\edsmdfaf}{D$_{\text{f}}$\textit{edsm}}
\newcommand{\edsmdfac}{D$_{\text{c}}$\textit{edsm}}
\newcommand{\hcdfaf}{D$_{\text{f}}$\textit{hc}}
\newcommand{\hcdfac}{D$_{\text{c}}$\textit{hc}}
\newcommand{\hcufaf}{U$_{\text{f}}$\textit{hc}}
\newcommand{\hcufac}{U$_{\text{c}}$\textit{hc}}
\newcommand{\delete}{DLT2}
\newcommand{\majvote}{MAJ}


\newcommand{\kcite}[1]{
\begin{verse}
\cite{#1}~\bibentry{#1}
\end{verse}
}
%\newcommand{\kcite}[1]{\cite{#1}}

%%%% debut macro pour faire des lignes �paisses dans les tableaux %%%%
\makeatletter
\def\hlinewd#1{%
\noalign{\ifnum0=`}\fi\hrule \@height #1 %
\futurelet\reserved@a\@xhline}
\makeatother
%%%% fin macro %%%%
